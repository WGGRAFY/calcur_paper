% Options for packages loaded elsewhere
\PassOptionsToPackage{unicode}{hyperref}
\PassOptionsToPackage{hyphens}{url}
%
\documentclass[
]{article}
\usepackage{amsmath,amssymb}
\usepackage{lmodern}
\usepackage{iftex}
\ifPDFTeX
  \usepackage[T1]{fontenc}
  \usepackage[utf8]{inputenc}
  \usepackage{textcomp} % provide euro and other symbols
\else % if luatex or xetex
  \usepackage{unicode-math}
  \defaultfontfeatures{Scale=MatchLowercase}
  \defaultfontfeatures[\rmfamily]{Ligatures=TeX,Scale=1}
\fi
% Use upquote if available, for straight quotes in verbatim environments
\IfFileExists{upquote.sty}{\usepackage{upquote}}{}
\IfFileExists{microtype.sty}{% use microtype if available
  \usepackage[]{microtype}
  \UseMicrotypeSet[protrusion]{basicmath} % disable protrusion for tt fonts
}{}
\makeatletter
\@ifundefined{KOMAClassName}{% if non-KOMA class
  \IfFileExists{parskip.sty}{%
    \usepackage{parskip}
  }{% else
    \setlength{\parindent}{0pt}
    \setlength{\parskip}{6pt plus 2pt minus 1pt}}
}{% if KOMA class
  \KOMAoptions{parskip=half}}
\makeatother
\usepackage{xcolor}
\IfFileExists{xurl.sty}{\usepackage{xurl}}{} % add URL line breaks if available
\IfFileExists{bookmark.sty}{\usepackage{bookmark}}{\usepackage{hyperref}}
\hypersetup{
  pdftitle={Modeling the effect of temperature on fish growth in the California Current},
  pdfauthor={Paul Spencer; Tara Marshall; Alan Boudron; Timothy J. Miller; Christine Stawitz; Melissa Haltuch},
  hidelinks,
  pdfcreator={LaTeX via pandoc}}
\urlstyle{same} % disable monospaced font for URLs
\usepackage[margin=1in]{geometry}
\usepackage{longtable,booktabs,array}
\usepackage{calc} % for calculating minipage widths
% Correct order of tables after \paragraph or \subparagraph
\usepackage{etoolbox}
\makeatletter
\patchcmd\longtable{\par}{\if@noskipsec\mbox{}\fi\par}{}{}
\makeatother
% Allow footnotes in longtable head/foot
\IfFileExists{footnotehyper.sty}{\usepackage{footnotehyper}}{\usepackage{footnote}}
\makesavenoteenv{longtable}
\usepackage{graphicx}
\makeatletter
\def\maxwidth{\ifdim\Gin@nat@width>\linewidth\linewidth\else\Gin@nat@width\fi}
\def\maxheight{\ifdim\Gin@nat@height>\textheight\textheight\else\Gin@nat@height\fi}
\makeatother
% Scale images if necessary, so that they will not overflow the page
% margins by default, and it is still possible to overwrite the defaults
% using explicit options in \includegraphics[width, height, ...]{}
\setkeys{Gin}{width=\maxwidth,height=\maxheight,keepaspectratio}
% Set default figure placement to htbp
\makeatletter
\def\fps@figure{htbp}
\makeatother
\setlength{\emergencystretch}{3em} % prevent overfull lines
\providecommand{\tightlist}{%
  \setlength{\itemsep}{0pt}\setlength{\parskip}{0pt}}
\setcounter{secnumdepth}{5}
\newlength{\cslhangindent}
\setlength{\cslhangindent}{1.5em}
\newlength{\csllabelwidth}
\setlength{\csllabelwidth}{3em}
\newlength{\cslentryspacingunit} % times entry-spacing
\setlength{\cslentryspacingunit}{\parskip}
\newenvironment{CSLReferences}[2] % #1 hanging-ident, #2 entry spacing
 {% don't indent paragraphs
  \setlength{\parindent}{0pt}
  % turn on hanging indent if param 1 is 1
  \ifodd #1
  \let\oldpar\par
  \def\par{\hangindent=\cslhangindent\oldpar}
  \fi
  % set entry spacing
  \setlength{\parskip}{#2\cslentryspacingunit}
 }%
 {}
\usepackage{calc}
\newcommand{\CSLBlock}[1]{#1\hfill\break}
\newcommand{\CSLLeftMargin}[1]{\parbox[t]{\csllabelwidth}{#1}}
\newcommand{\CSLRightInline}[1]{\parbox[t]{\linewidth - \csllabelwidth}{#1}\break}
\newcommand{\CSLIndent}[1]{\hspace{\cslhangindent}#1}
\usepackage{url}
\usepackage{setspace}
%\singlespacing
%\onehalfspacing
\doublespacing
\usepackage{lineno}
\linenumbers
\usepackage[belowskip=0pt,aboveskip=0pt]{caption}
\usepackage{relsize}
\newcommand{\afrb}{Alaska Fishery Research Bulletin\xspace}
\newcommand{\ajms}{African Journal of Marine Science\xspace}
\newcommand{\amb}{Advances in Marine Biology\xspace}
\newcommand{\bms}{Bulletin of Marine Science\xspace}
\newcommand{\bjssf}{Bulletin of the Japanese Society of Scientific Fisheries\xspace}
\newcommand{\cb}{Conservation Biology\xspace}
\newcommand{\cjfas}{Canadian Journal of Fisheries and Aquatic Sciences\xspace}
\newcommand{\ea}{Ecological Applications\xspace}
\newcommand{\eer}{Evolutionary Ecology Research\xspace}
\newcommand{\elet}{Ecology Letters\xspace}
\newcommand{\emod}{Ecological Modelling\xspace}
\newcommand{\ebf}{Environmental Biology of Fishes\xspace}
\newcommand{\ff}{Fish and Fisheries\xspace}
\newcommand{\fo}{Fisheries Oceanography\xspace}
\newcommand{\fr}{Fisheries Research\xspace}
\newcommand{\fb}{Fishery Bulletin\xspace}
\newcommand{\ijms}{ICES Journal of Marine Science\xspace}
\newcommand{\iccat}{Collective Volume of Scientific Papers ICCAT\xspace}
\newcommand{\jae}{Journal of Animal Ecology\xspace}
\newcommand{\jai}{Journal of Applied Ichthyology\xspace}
\newcommand{\jdc}{Journal Du Conseil International Pour L'exploration De La Mer\xspace}
\newcommand{\jdcp}{Journal Du Conseil Permanent International Pour L'exploration De La Mer\xspace}
\newcommand{\jembe}{Journal of Experimental Marine Biology and Ecology\xspace}
\newcommand{\jfb}{Journal of Fish Biology\xspace}
\newcommand{\jsr}{Journal of Sea Research\xspace}
\newcommand{\jtb}{Journal of Theoretical Biology\xspace}
\newcommand{\jfrbc}{Journal of the Fisheries Research Board of Canada\xspace}
\newcommand{\jnwafs}{Journal of Northwest Atlantic Fisheries Science\xspace}
\newcommand{\mcf}{Marine and Coastal Fisheries: Dynamics, Management, and Ecosystem Science\xspace}
\newcommand{\mb}{Marine Biology\xspace}
\newcommand{\meps}{Marine Ecology Progress Series\xspace}
\newcommand{\mfr}{Marine Fisheries Review\xspace}
\newcommand{\mpb}{Marine Pollution Bulletin\xspace}
\newcommand{\najfm}{North American Journal of Fisheries Management\xspace}
\newcommand{\nzjmfr}{New Zealand Journal of Marine and Freshwater Research\xspace}
\newcommand{\pnas}{Proceedings of the National Academy of Sciences USA\xspace}
\newcommand{\rpvrciemm}{Rapports et Proc\`es-Verbaux des R\'eunions. Conseil Internationale pour l'Exploration de la Mer\xspace}
\newcommand{\rpvrcpiemm}{Rapports et Proc\`es-Verbaux des R\'eunions. Conseil Permanent Internationale pour l'Exploration de la Mer\xspace}
\newcommand{\rfbf}{Reviews in Fish Biology and Fisheries\xspace}
\newcommand{\sajms}{South African Journal of Marine Science\xspace}
\newcommand{\tafs}{Transactions of the American Fisheries Society\xspace}

\newcommand{\anzjs}{Australian \& New Zealand Journal of Statistics\xspace}
\newcommand{\as}{Applied Statistics\xspace}
\newcommand{\csda}{Computational Statistics \& Data Analysis\xspace}
\newcommand{\ees}{Environmental and Ecological Statistics\xspace}
\newcommand{\jas}{Journal of Applied Statistics\xspace}
\newcommand{\jabes}{Journal of Agricultural, Biological, and Environmental Statistics\xspace}
\newcommand{\jasa}{Journal of the American Statistical Association\xspace}
\newcommand{\jrssb}{Journal of the Royal Statistical Society. Series B\xspace}
\newcommand{\sm}{Statistics in Medicine}

\usepackage{xspace}
\usepackage{bm}
\usepackage{caption,graphics}
\usepackage{graphicx}
\usepackage{makecell}
\renewcommand\figurename{Fig.}
\captionsetup{labelsep=period, singlelinecheck=false}
\usepackage{footmisc}
\newcommand{\changesize}[1]{\fontsize{#1pt}{#1pt}\selectfont}
\renewcommand{\arraystretch}{1.5}
\renewcommand\theadfont{}
\usepackage{booktabs}
\usepackage{longtable}
\usepackage{array}
\usepackage{multirow}
\usepackage{wrapfig}
\usepackage{float}
\usepackage{colortbl}
\usepackage{pdflscape}
\usepackage{tabu}
\usepackage{threeparttable}
\usepackage{threeparttablex}
\usepackage[normalem]{ulem}
\usepackage{makecell}
\usepackage{xcolor}
\ifLuaTeX
  \usepackage{selnolig}  % disable illegal ligatures
\fi

\title{Modeling the effect of temperature on fish growth in the
California Current}
\author{Paul Spencer\footnote{\href{mailto:paul.spencer@noaa.gov}{\nolinkurl{paul.spencer@noaa.gov}},
  Alaska Fisheries Science Center, National Marine Fisheries Service,
  7600 Sandpoint Way, Seattle, WA 98115, USA} \and Tara
Marshall\footnote{UK} \and Alan Boudron\footnote{UK} \and Timothy J.
Miller\footnote{Northeast Fisheries Science Center, National Marine
  Fisheries Service, 166 Water Street, Woods Hole, MA 02543, USA} \and Christine
Stawitz\footnote{Office of Science and Technology, National Marine
  Fisheries Service, 7600 Sandpoint Way, Seattle, WA 98115 USA} \and Melissa
Haltuch\footnote{Alaska Fisheries Science Center, National Marine
  Fisheries Service, 7600 Sandpoint Way, Seattle, WA 98115, USA}}
\date{}

\begin{document}
\maketitle

\pagebreak

\hypertarget{abstract}{%
\subsection*{Abstract}\label{abstract}}
\addcontentsline{toc}{subsection}{Abstract}

\hypertarget{keywords}{%
\subsubsection*{Keywords}\label{keywords}}
\addcontentsline{toc}{subsubsection}{Keywords}

growth; climate; stock assessment; reference points

\pagebreak

\hypertarget{introduction}{%
\section{Introduction}\label{introduction}}

The somatic growth of individual fish, from larval to adult stages,
underpins the size structuring of aquatic ecosystems and is also subject
to environmental influences (Black 2009). Growth is an inherently
non-linear process resulting from dynamic fluxes between anabolism and
catabolism (Quinn and Deriso 1999). Consequently, abrupt changes in
growth rates can occur throughout the lifespan of an individual fish,
most notably during the transition between the rapid growth of juvenile
stages and the slower growth during adult stages (Lester et al. 2004;
Quince et al. 2008). Dynamic changes in size-at-age can have direct
impacts on rates of harvest and fishery management reference points
because fisheries management is often based on limiting mortality (i.e.,
based on abundance) via biomass-based harvest quotas and the assumption
of an average growth relationship through time (Miller et al. 2018).
However, if growth fluctuates well above or below the long term average
growth relationship then realized biomass-based harvest quotas may be
lower or higher than quotas set as a function of an average growth
relationship. The dominant patterns in and magnitudes of somatic growth
variation over long time scales have not been quantified for many
commercially important fish stocks (Stawitz et al. 2015). Therefore,
quantifying the basic characteristics and dominant scales of variation
in growth is the necessary first step towards predicting growth
responses to biotic and abiotic factors (Stawitz et al. 2015).

Individual growth rates vary on a range of temporal and biological
scales, including across species and populations of the same species {[}
Brander (1995), Brunel and Dickey-Collas (2010)). Within a given system,
variation in size-at-age may occur between cohorts (intracohort) of a
single population Baudron et al. (2014), between years (annual changes),
or between juveniles of cohorts (initial size) (Stawitz et al. 2015).
Understanding the variation in size-at-age can help refine mechanistic
hypotheses. Density-independent annual changes in growth may occur from
processes such as upwelling, affecting all ages within a year.
Alternatively, density-dependent processes such as intracohort
competition may affect strong cohorts; in this case, environmental
processes that lead to strong recruitment may result in reduced growth
rates (Whitten et al. 2013). A third mechanism is that variability in
growth is related to only the size-at-age of juvenile fish, with growth
rates of older ages unaffected by the environment, which is consistent
with juvenile intracohort competition being the dominant process.
Stawitz et al. (2015) examined these hypotheses for North Pacific
groundfish, and found that about 40\% of the stocks studied showed
density-independent annual growth variation between years.

Because temperature is an important determinant of growth for
ectothermic species, the temperature size rule (TSR) provides the basis
for an important hypothesis relevant to climate change. The temperature
size rule (TSR) proposes that juvenile growth rates are higher in warmer
waters due to higher metabolic rates with rapid early growth leading to
a lower maximum (adult) size-at-age (Angilletta MJ Jr 2004; Daufresne
and Sommer 2009; Forster and Atkinson 2011; Forster and Hirst 2012). In
the context of warming regional seas, the TSR has the potential for
imposing a low-frequency signal into variability in individual growth
rates of fish. For example, warming temperatures in the North Sea
imposed a synchronous cross-species trend in growth rates of 6 of 8
commercial fish stocks consistent with the TSR (Baudron et al. 2014). A
combination of temperature-related reductions in body size and
distributional shifts has been estimated to reduce fisheries yields by
as much as 25\% (Cheung and Pauly 2013).

The von Bertalanffy growth function (VBGF; Bertalanffy (1938)),
developed from physiological concepts such as catabolism and anabolism
(Essington and Walters 2001) can be used to test how temperature may
affect size-at-age and potentially particular aspects (e.g., parameters)
of the growth process. Simple correlations between temperature and the
VBGF have been undertaken (Brunel and Dickey-Collas 2010), an approach
that would require temperature impacts to be strong relative to other
sources of variation. Adaptations of the VBGF have been developed to
incorporate the effect of temperature and other environmental factors
directly into the VBGF parameters \(L_\infty\) (asymptotic size) and k
(rate at which \(L_\infty\) is approached) (Fontoura and Agostinho 1996;
Shin and Rochet 1998). Kimura (2008) developed an extended form of the
VBGF which could include any explanatory variable as a covariate, which
Baudron et al. (2011) used to determine that temperature was a
statistically significant covariate in the cohort-specific VBGF fit for
North Sea haddock stock. Although the VBGF has the advantage of allowing
consideration of how environmental variability affects specific aspects
of growth, estimation of changes in both the k and \(L_\infty\)
parameters is difficult because they are highly correlated (Schnute and
Fournier 1980).

Rapid and variable local responses of fish size to warming can also make
temperature responses difficult to diagnose and predict at the ecosystem
scale (Audzijonyte 2020). Testing for a coherent (sensu consistent with
established physiology of ectotherms and widely observed across
different species) biological response to temperature at the ecosystem
scale requires using a statistical model suited to isolating the impacts
of a single, external factor (i.e., temperature) on fish growth rates in
addition to other possible sources of variation (e.g., density, prey
abundance, fisheries-induced changes in life history). Coherent and
synchronous annual growth trends across species that are consistent with
physiological principles (e.g., TSR) would imply there is a component of
growth variation that is a shared response to ecosystem-scale warming
(Baudron et al. 2014; Stawitz et al. 2015). Isolating such a response at
the stock- or ecosystem-level would provide the necessary empirical
support for models developed to forecast future fish yields (e.g.,
Cheung and Pauly (2013))

Because ecosystem observations do not come from controlled experiments
in which variables of interest can be isolated, more complex statistical
methods will be necessary to evaluate any potential coherent,
cross-species signal in the effect of temperature on growth. An
additional consideration is whether to model observation errors, which
is particularly relevant because the data available are typically
observations of size-at-age from individual fish (i.e., multiple
observations of size-at-age from a single fish that would more clearly
show individual growth are typically not available). Temporal trends in
size-at-age data could reflect trends in gear selectivity, sampling
locations, ageing bias and precision, and other factors. A closely
related concept is whether to employ Bayesian or random-effects methods
that would model observations and/or estimated parameters as random
variables. For example, changes in sampling location or the effect of
environmental covariates on size can be modeled as random variables to
account for unobserved heterogeneity not explained by the structural
model.

A variety of advanced statistical techniques have been applied recently
to evaluate variation in fish size-at-age. Baudron et al. (2014) used
Dynamic Factor Analysis (DFA; Zuur et al. (2003)) to estimate common
``latent'' trends in the cohort-specific \(L_\infty\) time series for
eight North Sea stocks with long time series of size-at-age. DFA is a
multivariate extension of structural time series, with the time series
for a particular stock being a function of underlying latent trends and
stock-specific observation error. An alternative framework applied by
Stawitz et al. (2015) is an autoregressive state-space model consisting
of process and observation models that fit to observed time series of
standardized length-at-age data without a mechanistic growth model.
Miller et al. (2018) also used a state-space model, but the process
model is based on a generalized VBGF that allowed process errors in the
k parameter. State-space models simultaneously estimate model parameters
using two equations: the autoregressive process representing abiotic and
biotic covariates and the unobserved processes including space and time
covariates.

The inferences that can be made regarding how temperature affects fish
growth are influenced by the choice of model structure. In particular,
evaluation of a series of models may help illuminate modeling approaches
that have utility when any coherent response of fish growth to
temperature may be subtle relative to asynchronous or stock-specific
factors (e.g., food availability, density). Although the models
mentioned above all have a common property of recognizing variation
other than the ``process'' variation of interest, they apportion
variance very differently because of differences in model structure.
Multi-model inference is the process whereby the response variable is
estimated using several candidate models rather than a single `best'
model (Burnham and Anderson 2002) and has been previously applied to
growth modelling (Katsanevakis and Maravelias 2008). In our study the
intent was not necessarily to predict the response variable with the
greatest accuracy but to identify the strengths and weaknesses of
alternative modelling framework(s) for assessing whether there was a
synchronous impact of temperature once other sources of variation have
been accounted for (state-space models) or once asynchronous sources of
variation had been excluded (DFA). Comparative analysis of models can
also help to identify biases in model performance (e.g.~whether a model
systematically underestimates random noise in the data) or shortcomings
in model fitting (e.g., estimation of process or observation error). For
example, Brodie and Selden (2020) compared several types of species
distribution models and identified which models were appropriate for
specific purposes. Ultimately model purpose must be given due
consideration when deciding on the most appropriate modelling framework
(Guillera-Arroita and Wintle 2015; Brodie and Selden 2020).

The aim of this study is to undertake a comparative analysis of four
models using empirical data for temperature and size-at-age for seven
fish stocks in the California Current large marine ecosystem. A common
metric for the models is to evaluate the degree to which size-at-age is
related to temperature. Estimates of uncertainties, parameter
correlations, and any potential aliasing (i.e., erroneously attributing
variation size-at-age to non-causal mechanisms or covariates). We
consider the relative strength and weakness of the models in describing
how temperature may affect size-at-age, and how the combination of
models can be used for multi-model inference.

\hypertarget{methods}{%
\section{Methods}\label{methods}}

\hypertarget{study-area-species-chosen-trawl-survey-description-melissa}{%
\subsection{Study area, species chosen, trawl survey description
(Melissa)}\label{study-area-species-chosen-trawl-survey-description-melissa}}

Four trawl surveys conducted by the National Marine Fisheries Center's
Alaska Fisheries Science Center (AFSC) and Northwest Fisheries Science
Center (NWFSC) provided data for this study. The Triennial Shelf Survey,
conducted by the AFSC in 1980, 1983, 1986, 1989, 1992, 1995, 1998, and
2001 and by the NWFSC in 2004, provides the earliest time series of
fishery independent temperature and biological data in the U.S. portion
of the California Current continental shelf (Weinberg et al. 2002).
Triennial survey sampling occurred along transects perpendicular to the
coast over depths from 55 m to 366 m (500 m after 1992) and from the
Canadian border to Monterey Bay, California (36°48' N) until 1986, then
to Point Conception, California (34°30' N) from 1989 forward.

The transect-based AFSC Slope Survey was conducted in 1997, 1999, 2000,
and 2001, over depths from 184 m to 1,280 m in waters north of Point
Conception, California (34°30' N) to the US -- Canada border (Lauth
1999, 2000, 2001). Sampling in earlier years was spatially limited,
covering small and inconsistent portions of the coast; however, this
survey had a high degree of biological sampling.

The transect-based NWFSC Slope survey, conducted in 1998, 1999, 2000,
2001, and 2002, covered depths ranging from 184 m to 1,280 m using
chartered commercial fishing vessels \textless93 feet (Keller et al.
2017). Prior to 2000 the NWFSC Slope survey sampled from the Morro Bay,
California (lat 35°00'N), to the U.S.--Canada border, the survey area
was expanded south to Point Conception, California (34°30' N) in 2001,
then to the U.S. Mexico border in 2002. This survey consists of fewer
tows compared to other survey data sets, with a lower fraction of tows
sampled for ages.

The NWFSC West Coast Groundfish Bottom Trawl Survey (WCGBTS), operating
annually from 2003 through 2019, implements a stratified random-grid
survey design that spans both continental shelf and slope habitats,
depths from 55m to 1,280 m, and covers U.S. waters between the Canada
and Mexico borders (Bradburn et al. 2011; Keller et al. 2017). Strata
include three depth strata (55 m to 183 m, 184 m to 549 m, and 550 m
to1280 m) and two spatial strata (north and south of Point Conception,
California (34°30' N)) for all years except 2003. In 2003 five spatial
strata delineated by the boundaries of the International North Pacific
Fisheries Commission (INPFC) statistical areas were used. These INPFC
statistical areas are, from north to south, Vancouver, Columbia, Eureka,
Monterey, and Conception. Generally, four chartered industry vessels
conduct tows from late-May to early-October, in randomly selected grid
cells, during two north to south passes along the U.S. west coast.
Randomly sampled lengths and ages are collected; age structures are
sampled from a subset of the fish that have been measured for length.
Major changes in the WCGBTS, compared to prior surveys, include
implementing a stratified random survey design and consistent spatial
coverage south of Point Conception, California (34°30' N).

Regions used in this study are a combination of the 2003 WCGBTS
latitudinal and depth strata. The INPFC Vancouver, Columbia, and Eureka
latitudinal strata are combined into a single region, labeled ``ECV'',
due to ecological similarity. The combination of three INPFC latitudinal
strata and three depth strata yield nine regions defined by latitude and
depth. Mean bottom temperature used in situ data from the four surveys
for each of the nine region-year strata with at least five observations,
with data from the AFSC slope survey restricted to September -- October
to provide a similar sampling period to the remaining surveys (i.e., May
-- October).

Seven species were selected for analysis of length-at-age data, based on
a diversity of life-history traits, habitat usage, and data
availability. Length-at-age data for two deep-water species with ranges
encompassing both the continental shelf and slope (darkbloched rockfish
(\emph{Sebastes crameri}) and sablefish (\emph{Anoplopoma fimbria}))
were obtained from two surveys that collectively cover the full depth
range of these species in a given year: the Triennial Shelf Survey and
NWFSC Slope surveys for years 1998 and 2001, and the WCGBTS for the
years 2003 to 2018. Additionally, samples from the 2001 AFSC slope
survey, and the 2004 Triennial shelf survey, were used for these two
species. Shortbelly rockfish (\emph{Sebastes jordani}) also occur on the
slope and shelf, but limited observations restricted analysis to data
from the WCGBTS. Length-at-age data for four species that occur on the
continental shelf (Pacific hake (\emph{Merluccius productus}), Pacific
sanddab (\emph{Citharichthys sordidus}), lingcod (\emph{Ophiodon
elongatus}), and petrale sole (\emph{Eopsetta jordani})) were obtained
from two surveys: the Triennial Shelf Survey and the WCGBTS. Because
marine fish stocks often change the depth and spatial locations occupied
as they age, we attempted to derive temperature time series reflective
of the habitats they occupy at a given age. We used the species-specific
set of surveys listed above to compute the mean depth and latitude of
the length-at-age samples by age (for ages 0 -- 15) across years, then
classified these means into the nine latitude-depth regions. The
temperature time series for a given species and age was obtained from
the corresponding latitude-depth region where they occurred.

\hypertarget{state-space-size-at-age-model}{%
\subsection{State-space size-at-age
model}\label{state-space-size-at-age-model}}

We used a slight modification of the state-space size-at-age model in
Stawitz et al. (2015). Mean length-at-age for each year for each species
is calculated for fish ages with at least ten observations across the
length of the time series. The mean is calculated as the difference in
size-at-age for a given age and year from the mean size-at-age for that
age across a 15 year reference period (1995 - 2010) to allow for
comparisons between species with different time series lengths.

This model assumes an autoregressive process of order 1 (AR1) represents
the ``true'' temporal processes in size-at-age data, such that the
size-at-age \(a+1\) and year \(t+1\) is a function of size-at-age \(a\)
in year \(t\). This process can be modeled as an AR1 process only
(\(S_1\)) or with four additional predictors that represent
year-specific (\(S_2\)), cohort-specific (\(S_3\)), or initial size
(\(S_4\)) deviations from the AR1 process shared across age groups, or
all three types of deviations (\(S_5\)) (Table 3). An additional three
models include a temperature covariate added to the annual variation
model (\(S_6\)), to the cohort variation model (\(S_7\)) or to the
initial size variation model (\(S_8\)). The annual temperature model
(\(S_6\)) uses the temperature at time \(t-1\) to predict size in year
\(t\). The cohort temperature model (\(S_7\)) and initial size
temperature model (\(S_8\)) uses temperature in the first year of life,
which is zero for all species with the exception of petrale sole, who
have a first age of one. The effect of using the same cohort temperature
in both \(S_7\) and \(S_8\) is to test if the temperature at the birth
year affects growth only in the first year of life or in subsequent
years.

The observation equation is the same across all models, such that the
observed length-at-age is normally-distributed around the ``true''
length-at-age. The priors and estimated parameter values for this model
are provided in Appendix . The model is implemented in the Stan modeling
language (Carpenter (2017)) via the \texttt{sarla} (State-space
AutoRegressive Length-at-Age) \texttt{R} package which can be downloaded
at \url{http://github.com/wggrafy/sarla}. Model convergence was assessed
by visual examination of trace plots, posterior estimates, and posterior
predictive interval checks against the data. E-BFMI and R-hat statistics
provided by Stan were also satisfactory. We used the approximate
leave-future-out cross-validation statistic presented in Bürkner et al.
(2020) to perform model selection.

\hypertarget{state-space-von-bertalanffy-growth-models}{%
\subsection{State-space von Bertalanffy growth
models}\label{state-space-von-bertalanffy-growth-models}}

We used the same general model as Miller et al. (2018) which assumes von
Bertalanffy growth and allows annual age-specific growth rates. It can
use combinations of length and weight information and simultaneously
estimates allometric length-weight relationship. We fit four alternative
models for each species (Table 1). The base model (\(G_1\)) assumes the
same LVB growth rate and asymptotic size for all individuals. The second
model (\(G_2\)) allows annual AR(1) deviations in the log growth rate
for a given year applied to all cohorts in that year. Allowing this
AR(1) process for annual growth rates was found by Miller et al. (2018)
to be important for evaluating effects of temperature on growth rates
for Georges Bank Atlantic cod. The third model (\(G_3\)) expands the
second model to include temperature effects on the growth rate during
the first year of life. The fourth model (\(G_4\)) expands the third
model to allow different asymptotic lengths for cohorts originating
after 2000 when there was a dramatic reduction in fishing pressure
(Warlick et al. 2018).

For models \(G_3\) and \(G_4\), we used the bottom temperature estimates
for the region defined by area and depth bin where the species was
predominantly found during the earliest observed ages (Table 2). We
observed age 0 fish for all species except petrale sole (minimum age is
1), and the youngest observed ages of all species predominated in depths
less than 184 m. Young sablefish and darkblotched rockfish predominated
in the most northern region (Eureka-Columbia-Vancouver) and young
shortbelly rockfish predominated in the most southern region
(Conception). Young fish of all other species predominated in the
intermediate region (Monterey).

\hypertarget{dynamic-factor-analysis}{%
\subsection{Dynamic Factor Analysis}\label{dynamic-factor-analysis}}

The Dynamic Factor Analysis (DFA) (Zuur et al. 2003) is a multivariate
time series analysis which allows estimating underlying common trend(s)
among a set of time series. DFA aims at modelling as few common trends
as possible whilst still achieving a reasonable model fit, and has
previously been used successfully to describe temporal variation in size
among species (Zuur et al. 2003; Baudron et al. 2014; Goertler 2016;
Ikpewe et al. 2021). The size of a species \(s\) in year \(t\) is
modeled as follows:

\[
size_s(t)= Z_{1,s}x_{1,t}+\dots+Z_{i,s}x_{i,t}+ a_s+\varepsilon_{s,t} 
\]

where \(i\) is the number of common trends \(x\) identified in the data,
\(Z\) is the species-specific factor loading indicating how a common
trend is related to the original time series for that species, a is an
offset term, and \(\varepsilon_t \sim MVN(0, \textbf{R})\) with MVN
standing for Multivariate Normal and R for the error covariance matrix.

The DFA model isn't spatially explicit but instead was applied
independently in each of the three areas i.e., ECV, Monterey and
Conception, to investigate growth patterns across the seven species. Two
different methods were used: the first follows that of Baudron et al.
(2014), while the second follows that of Ikpewe et al. (2021). For the
first method, a von Bertalanffy growth model was fitted on a cohort
basis (assuming individuals within a cohort exhibit similar growth
trajectories) to each species in each area. Only \(L_{\infty}\) values
that were significantly estimated (\(P<0.05\)) were considered, and
\(L_{\infty}\) time series were standardized by subtracting the mean and
dividing by the standard deviation. The DFA was then used to identified
common trends in \(L_{\infty}\) time series among species in each area.
For the second method, the age at which 50\% of individuals are mature
(\(A50\)) was first obtained from the literature for each species (see
Supplementary material) and considered as the maturing age. Then, the
juvenile age was defined as \(A50/2\), and the mature age as
\((\mathrm{max age} – A50)/2\), where max age is the maximum age
observed in the survey data for each species. Time series of mean length
at juvenile age, maturing age, and mature age (i.e., three time series
per species) were computed for each species in each area and
standardized by subtracting the mean and dividing by the standard
deviation. The DFA was then used to identified common trends in mean
length time series among species in each area for the three life stage
selected: juvenile, maturing and mature.

In both methods one and two, the DFA models in each area were run with
and without covariates and the best model in each area was identified as
the one with the lowest AIC. Two covariates were considered: temperature
anomaly, and fishing pressure. Temperature anomaly was estimated for
depths greater than 550 m (ECV), depths between 184 and 550 m (ECV and
Monterey), and depths less than 183 m (ECV, Monterey and Conception).
Fish stocks in the California current ecosystem were all subject to high
fishing pressure (F) until 2000, after which it was drastically reduced
(Richards 2017). To mimic this, a dummy F time series was computed with
\(F \sim 0.8\) from 1977 to 2000, and \(F \sim 0.2\) from 2001 to 2018.
Random noise around the values 0.2 and 0.8 was added with the ``jitter''
function in R to make the F time series more realistic. DFA models were
run with temperature and F separately, and with both covariates
combined. All DFA models were run using the MARSS (Holmes et al. 2021) R
package (version 3.11.4).

\hypertarget{results}{%
\section{Results}\label{results}}

\hypertarget{description-of-the-size-at-age-data-and-temperature-data-paul}{%
\subsection{Description of the size-at-age data, and temperature data
(Paul)}\label{description-of-the-size-at-age-data-and-temperature-data-paul}}

The number of samples, and length of time series, varied by species,
owing to the different habitats and sampling rates for the various
surveys. Pacific Hake has samples extending back to 1977, with large
samples in the early years of the Triennial shelf survey. In contrast,
the time series for Pacific sanddab, Petrale sole, and shortbelly
rockfish begins in 2003 with the WCGBTS. The most samples were obtained
for sablefish (23,470), whereas the least samples were obtained for
shortbelly rockfish (5,990).

Observed temperatures decreased with depth and latitude (Figure x). At
the shallowest depth strata of \textless{} 183 m, the average
temperature from 2014-2018 in the northern ECV region was 7.7 C; in
contrast, the average in the southern Conception area was 10.2 C. The
pattern of higher temperatures in the southern regions was observed
across all depth strata, although in the deepest depth strata of
\textgreater{} 550 m the temperature differences across the latitude
regions were less distinct. A drop in temperatures was observed in 1999
in the two deepest depth strata of 184 - 550 m and \textgreater{} 550 m,
corresponding to the 1999 La Nina event (ref). In the shallowest depth
strata of \textless{} 183 m, increases in temperatures were observed
between 2013 and 2014/2015. In general, however, the temperatures
observed from the surveys have been generally stable with long-term
directional trends.

\hypertarget{results-from-the-three-models}{%
\subsection{Results from the three
models}\label{results-from-the-three-models}}

\hypertarget{stawitz-state-space-model-christine}{%
\subsubsection{Stawitz state-space model
(Christine)}\label{stawitz-state-space-model-christine}}

\hypertarget{state-space-von-bertalanffy-growth-model-tim}{%
\subsubsection{State-space von Bertalanffy growth model
(Tim)}\label{state-space-von-bertalanffy-growth-model-tim}}

We found that including temperature effects on growth during the first
year of life improves model performance for all 7 species (Table 3). The
estimated temperature effect was positive for all species except
shortbelly rockfish (Table 4, Figure 1), the species with the most
southerly distribution. Allowing asymptotic size to differ for cohorts
exposed to low or high fishing pressure improved model performance for 4
of the 7 species (darkblotched rockfish, Pacific hake, sablefish, and
shortbelly rockfish). Estimates of asymptotic size for darkblotched
rockfish and Pacific hake were greater after fishing pressure was
released, but estimates were lower for sablefish and shortbelly rockfish
(Table 5).

\hypertarget{vb-dba-alan}{%
\subsubsection{VB-DBA (Alan)}\label{vb-dba-alan}}

With the first method (DFA applied to \(L_{\infty}\) time series), the
DFA models identified a single and declining common trend in all three
areas, consistent with TSR (Fig. 2). With the exception of lingcod, the
\(L_{\infty}\) time series for all species were positively related to
the common trends in all three areas, as shown by the positive factor
loadings. The common trend was best supported by the data in the
Monterey area, where five out of seven species showed similar factor
loading values indicating equal support for the identified trend from a
majority of species. In both ECV and Conception areas the identified
trends were supported by three species only. Including covariates did
not improve the DFA models (Table 6). For both Monterey and Conception
areas the lowest AIC was achieved by the model with no covariates. For
the ECV area, the model including \(F\) achieved the lowest AIC, however
the difference with the AIC from the model without covariates was less
than 4 indicating that the inclusion of the \(F\) covariate does not
significantly improve the model (Burnham and Anderson 2002).

With the second method (DFA applied to mean length at juvenile, maturing
and mature life stages), the DFA models did not indicate a clear
directional pattern across areas and life stages (Fig. S?). However, in
the Monterey area where the common declining trend in \(L_{\infty}\) was
best supported, there was some indication of an increase in the mean
length of juveniles coinciding with a decline in the mean length of
mature individuals, at least in the second part of the time series
(mid-1990s onwards), consistent with TSR (Fig. 3). Unfortunately, the
factor loadings showed poor support across species for both these
trends, with some species being highly negatively correlated to the
trends identified. As with the first method, including covariates did
not improve the DFA models: in all three areas and for all three life
stages, the best DFA model was always the one without covariates (Table
7).

\hypertarget{discussion}{%
\section{Discussion}\label{discussion}}

To be determined, but some organizing thoughts are:

\begin{enumerate}
\def\labelenumi{\arabic{enumi})}
\item
  Potential for attribution (and maybe misattribution) of sources of
  variability with differing modeling approaches (i.e., spatial
  vs.~non-spatial models)
\item
  Advantages/disadvantages of mechanistic models vs non-mechanistic
  models
\item
  The ability of the modeling approaches to be able to distinguish
  between the hypotheses of fish growth
\item
  The role of biology in affecting the influence of temperature on
  size-at-age (i.e., ontogenetic depth movement, timing of spawning
\item
  Finally, spatio-temporal models such as VAST (Vector Autoregressive
  Spatio-Temporal model; www.github.com/james-thorson/VAST) are
  mixed-effect models in which model spatial variation as random effects
  given a pattern of spatial correlation, and a number of covariates can
  be modeled as fixed effects. Although VAST models are often applied to
  data on fish density from resource surveys (Thorson 2019), they can be
  potentially useful for cases where fish size-at-age may vary over
  space in patterns not related to the modeled fixed effects.
\end{enumerate}

\hypertarget{conclusion}{%
\section{Conclusion}\label{conclusion}}

\hypertarget{acknowledgements}{%
\section*{Acknowledgements}\label{acknowledgements}}
\addcontentsline{toc}{section}{Acknowledgements}

\pagebreak

\hypertarget{references}{%
\subsection*{References}\label{references}}
\addcontentsline{toc}{subsection}{References}

\hypertarget{refs}{}
\begin{CSLReferences}{1}{0}
\leavevmode\vadjust pre{\hypertarget{ref-angilletta04}{}}%
Angilletta MJ Jr, S.M., Steury TD. 2004. Temperature, growth rate, and
body size in ectotherms: Fitting pieces of a life-history puzzle. Integr
Comp Biol. \textbf{44}: 498--509.
doi:\href{https://doi.org/10.1093/icb/44.6.498}{10.1093/icb/44.6.498}.

\leavevmode\vadjust pre{\hypertarget{ref-audzijonyte20}{}}%
Audzijonyte, R., A. 2020. Fish body sizes change with temperature but
not all species shrink with warming. Nature ecology \& evolution
\textbf{4}: 809-\/-814.
doi:\href{https://doi.org/10.1038/s41559-020-1171-0}{10.1038/s41559-020-1171-0}.

\leavevmode\vadjust pre{\hypertarget{ref-baudronetal11}{}}%
Baudron, A.R., Needle, C.L., and Marshall, C.T. 2011. Implications of a
warming {N}orth {S}ea for the growth of haddock \emph{{M}elanogrammus
aeglefinus}.\jfb \textbf{78}(7): 1874--1889.

\leavevmode\vadjust pre{\hypertarget{ref-baudron14}{}}%
Baudron, A.R., Needle, C.L., Rijnsdorp, A.D., and Marshall, C.T. 2014.
Warming temperatures and smaller body sizes: Synchronous changes in
growth of north sea fishes. Global Change Biology \textbf{20}(4):
1023--1031.

\leavevmode\vadjust pre{\hypertarget{ref-vonbert38}{}}%
Bertalanffy, L. von. 1938. A quantitative theory of organic growth.
Human Biology \textbf{10}: 181--213.

\leavevmode\vadjust pre{\hypertarget{ref-black09}{}}%
Black, B.A. 2009. {Climate-driven synchrony across tree, bivalve, and
rockfish growth-increment chronologies of the northeast Pacific}.\meps
\textbf{378}: 37--46.
doi:\href{https://doi.org/10.3354/meps07854}{10.3354/meps07854}.

\leavevmode\vadjust pre{\hypertarget{ref-bradburnetal11}{}}%
Bradburn, M., Keller, A., and Horness, B. 2011. {The 2003 to 2008 U.S.
West Coast bottom trawl surveys of groundfish resources off Washington,
Oregon, and California: estimates of distribution, abundance, length,
and age composition}. {Tech. Rep. NMFS-NWFSC-114, U.S. Department of
Commerce, Seattle, WA}.

\leavevmode\vadjust pre{\hypertarget{ref-brander95}{}}%
Brander, K.M. 1995. The effect of temperature on growth of atlantic cod
(gadus morhua). ICES Journal of Marine Science \textbf{52}: 1--10.

\leavevmode\vadjust pre{\hypertarget{ref-brodie20}{}}%
Brodie, T., S. J., and Selden, R.L. 2020. Trade‐offs in covariate
selection for species distribution models: A methodological comparison.
Ecography \textbf{43}: 11--24.

\leavevmode\vadjust pre{\hypertarget{ref-brunel10}{}}%
Brunel, T., and Dickey-Collas, M. 2010. Effects of temperature and
population density on von bertalanffy growth parameters in atlantic
herring: A macro-ecological analysis.\meps \textbf{405}: 15--28.

\leavevmode\vadjust pre{\hypertarget{ref-burkner20}{}}%
Bürkner, P.-C., Gabry, J., and Vehtari, A. 2020. Approximate
leave-future-out cross-validation for bayesian time series models.
Journal of Statistical Computation and Simulation \textbf{90}(14):
2499--2523. Informa {UK} Limited.
doi:\href{https://doi.org/10.1080/00949655.2020.1783262}{10.1080/00949655.2020.1783262}.

\leavevmode\vadjust pre{\hypertarget{ref-burnhamanderson02}{}}%
Burnham, K.P., and Anderson, D.R. 2002. Model selection and multimodel
inference: A practical information-theoretic approach. Springer-Verlag,
New York.

\leavevmode\vadjust pre{\hypertarget{ref-stan}{}}%
Carpenter, G., B. 2017. Stan: A probabilistic programming language.
Journal of Statistical Software \textbf{76}(1): 1--32.
doi:\url{https://doi.org/10.18637/jss.v076.i01}.

\leavevmode\vadjust pre{\hypertarget{ref-cheung13}{}}%
Cheung, S., W. W. L., and Pauly, D. 2013. Shrinking of fishes
exacerbates impacts of global ocean changes on marine ecosystems. Nature
Climate Change \textbf{3}: 254--258.

\leavevmode\vadjust pre{\hypertarget{ref-daufresne}{}}%
Daufresne, L., M., and Sommer, U. 2009. Global warming benefits the
small in aquatic ecosystems. Proc Natl Acad Sci \textbf{106}:
12788--12793.

\leavevmode\vadjust pre{\hypertarget{ref-essington01}{}}%
Essington, K., T. E., and Walters, C.J. 2001. The von bertalanffy
grovuth function, bioenergetics, and the consumption rates of fish.
Canadian Journal of Fisheries and Aquatic Sciences \textbf{58}:
2129--2138.

\leavevmode\vadjust pre{\hypertarget{ref-fontoura}{}}%
Fontoura, N. F., and Agostinho, A.A. 1996. Growth with seasonally
varying temperatures: An expansion of the von bertalanffy growth model.
Journal of Fish Biology: 569--584.

\leavevmode\vadjust pre{\hypertarget{ref-forster12}{}}%
Forster, H., J., and Atkinson, D. 2011. How do organisms change size
with changing temperature? The importance of reproductive method and
ontogenetic timing. Functional Ecology \textbf{25}: 1024--1031.

\leavevmode\vadjust pre{\hypertarget{ref-forster11}{}}%
Forster, J., and Hirst, A.G. 2012. The temperature-size rule emerges
from ontogenetic differences between growth and development rates.
Functional Ecology \textbf{26}: 483--492.

\leavevmode\vadjust pre{\hypertarget{ref-goertler16}{}}%
Goertler, M.D.A.S., Pascale A. L. AND Scheuerell. 2016. Estimating
common growth patterns in juvenile chinook salmon (oncorhynchus
tshawytscha) from diverse genetic stocks and a large spatial extent.
PLOS ONE \textbf{11}(10): 1--19. Public Library of Science.
doi:\href{https://doi.org/10.1371/journal.pone.0162121}{10.1371/journal.pone.0162121}.

\leavevmode\vadjust pre{\hypertarget{ref-Guillera-Arroita15}{}}%
Guillera-Arroita, L.-M., G., and Wintle, B.A. 2015. Is my species
distribution model fit for purpose? Matching data and models to
applications. Global Ecology and Biogeography \textbf{24}(3): 276--292.

\leavevmode\vadjust pre{\hypertarget{ref-holmes21}{}}%
Holmes, E.E., Ward, E.J., and Scheuerell, M.D. 2021, December. {Analysis
of multivariate time series using the MARSS package. Version 3.11.4}.
Zenodo.
doi:\href{https://doi.org/10.5281/zenodo.5781847}{10.5281/zenodo.5781847}.

\leavevmode\vadjust pre{\hypertarget{ref-ikpewe21}{}}%
Ikpewe, I.E., Baudron, A.R., Ponchon, A., and Fernandes, P.G. 2021.
Bigger juveniles and smaller adults: Changes in fish size correlate with
warming seas. Journal of Applied Ecology \textbf{58}(4): 847--856.
doi:\url{https://doi.org/10.1111/1365-2664.13807}.

\leavevmode\vadjust pre{\hypertarget{ref-katsanevakis08}{}}%
Katsanevakis, S., and Maravelias, C.D. 2008. Modelling fish growth:
Multi-model inference as a better alternative to a priori using von
bertalanffy equation. Fish and Fisheries \textbf{9}(2): 178--187.
doi:\href{https://doi.org/10.1111/j.1467-2979.2008.00279.x}{10.1111/j.1467-2979.2008.00279.x}.

\leavevmode\vadjust pre{\hypertarget{ref-kelleretal17}{}}%
Keller, A.A., Wallace, J.R., and Methot, R.D. 2017. The {Northwest
Fisheries Science Center's} west coast groundfish bottom trawl survey:
History, design, and description. Seattle, WA: NOAA.

\leavevmode\vadjust pre{\hypertarget{ref-kimura08}{}}%
Kimura, D.K. 2008. Extending the von bertalanffy growth model using
explanatory variables. Canadian Journal of Fisheries and Aquatic
Sciences \textbf{65}(9): 1879--1891.

\leavevmode\vadjust pre{\hypertarget{ref-lauth99}{}}%
Lauth, R.R. 1999. The 1997 pacific west coast upper continental slope
trawl survey of groundfish resources off washington, oregon, and
california: Estimates of distribution, abundance, and length
composition. {U.S. Department of Commerce, NOAA Technical Memorandum
NMFSAFSC-98}.

\leavevmode\vadjust pre{\hypertarget{ref-lauth00}{}}%
Lauth, R.R. 2000. {The 1999 Pacific West Coast Upper Continental Slope
Trawl Survey of groundfish resources off Washington, Oregon, and
California: Estimates of distribution, abundance, and length
composition}. {U.S. Department of Commerce, NOAA Technical Memorandum
NMFSAFSC-115}.

\leavevmode\vadjust pre{\hypertarget{ref-lauth01}{}}%
Lauth, R.R. 2001. {The 2000 Pacific West Coast Upper Continental Slope
Trawl Survey of groundfish resources off Washington, Oregon, and
California: Estimates of distribution, abundance, and length
composition}. {U.S. Department of Commerce, NOAA Technical Memorandum
NMFSAFSC-120}.

\leavevmode\vadjust pre{\hypertarget{ref-lesteretal04}{}}%
Lester, N.P., Shuter, B.J., and Abrams, P.A. 2004. Interpreting the von
bertalanffy model of somatic growth in fishes: The cost of reproduction.
Proceedings of the Royal Society of London. Series B: Biological
Sciences \textbf{271}(1548): 1625--1631.
doi:\href{https://doi.org/10.1098/rspb.2004.2778}{10.1098/rspb.2004.2778}.

\leavevmode\vadjust pre{\hypertarget{ref-milleretal18}{}}%
Miller, T.J., O'Brien, L., and Fratantoni, P.S. 2018. Temporal and
environmental variation in growth and maturity and effects on management
reference points of {G}eorges {B}ank {A}tlantic cod. Canadian Journal of
Fisheries and Aquatic Sciences \textbf{75}(12): 2159--2171.
doi:\href{https://doi.org/10.1139/cjfas-2017-0124}{10.1139/cjfas-2017-0124}.

\leavevmode\vadjust pre{\hypertarget{ref-quinceetal08}{}}%
Quince, C., Abrams, P.A., Shuter, B.J., and Lester, N.P. 2008. Biphasic
growth in fish i: Theoretical foundations. Journal of Theoretical
Biology \textbf{254}(2): 197--206.
doi:\href{https://doi.org/10.1016/j.jtbi.2008.05.029}{10.1016/j.jtbi.2008.05.029}.

\leavevmode\vadjust pre{\hypertarget{ref-quinnderiso99}{}}%
Quinn, T.J., and Deriso, R.B. 1999. Quantitative fish dynamics. Oxford
University Press.

\leavevmode\vadjust pre{\hypertarget{ref-richards17}{}}%
Richards, M. 2017. From disaster to sustainability: The story of the
pacific groundfish. PhD thesis, Utah State University, Logan, UT.
doi:\href{https://doi.org/10.26076/1947-80f4}{10.26076/1947-80f4}.

\leavevmode\vadjust pre{\hypertarget{ref-schnutefournier}{}}%
Schnute, J., and Fournier, D. 1980. A new approach to length-frequency
analysis: Growth structure. Canadian Journal of Fisheries and Aquatic
Sciences \textbf{39}: 1337--1351.

\leavevmode\vadjust pre{\hypertarget{ref-shin}{}}%
Shin, Y.-J., and Rochet, M.-J. 1998. A model for the phenotypic
plasticity of north sea herring growth in relation to trophic
conditions. Aquatic Living Resources \textbf{11}: 315--324.

\leavevmode\vadjust pre{\hypertarget{ref-stawitzetal15}{}}%
Stawitz, C.C., Essington, T.E., Branch, T.A., Haltuch, M.A., Hollowed,
A.B., and Spencer, P.D. 2015. A state-space approach for detecting
growth variation and application to {N}orth {P}acific groundfish.\cjfas
\textbf{72}(9): 1316--1328.

\leavevmode\vadjust pre{\hypertarget{ref-warlicketal18}{}}%
Warlick, A., Steiner, E., and Guldin, M. 2018. {History of the West
Coast groundfish trawl fishery: Tracking socioeconomic characteristics
across different management policies in a multispecies fishery}. Marine
Policy \textbf{93}: 9--21.

\leavevmode\vadjust pre{\hypertarget{ref-weinbergetal02}{}}%
Weinberg, K.L., Wilkins, M.E., Shaw, F.R., and Zimmermann, M. 2002. The
2001 {P}acific {W}est {C}oast bottom trawl survey of groundfish
resources: Estimates of distribution, abundance, and length and age
composition. {U}.{S}. {D}epartment of {C}ommerce, {NOAA} {T}echnical
{M}emorandum {NMFS-AFSC}-128.

\leavevmode\vadjust pre{\hypertarget{ref-whitten13}{}}%
Whitten, A.R., Klaer, N.L., Tuck, G.N., and Day, R.W. 2013. Accounting
for cohort-specific variable growth in fisheries stock assessments: A
case study from south-eastern australia. Fisheries Research
\textbf{142}: 27--36.

\leavevmode\vadjust pre{\hypertarget{ref-zuur03}{}}%
Zuur, A.F., Fryer, R.J., Jolliffe, I.T., Dekker, R., and Beukema, J.J.
2003. Estimating common trends in multivariate time series using dynamic
factor analysis. Environmetrics: 665--685.

\end{CSLReferences}

\pagebreak

\hypertarget{appendix-a}{%
\section*{Appendix A}\label{appendix-a}}
\addcontentsline{toc}{section}{Appendix A}

\hypertarget{tables}{%
\section{Tables}\label{tables}}

Table 3: State space size-at-age models

\begin{longtable}[]{@{}
  >{\raggedright\arraybackslash}p{(\columnwidth - 4\tabcolsep) * \real{0.0933}}
  >{\raggedright\arraybackslash}p{(\columnwidth - 4\tabcolsep) * \real{0.5667}}
  >{\raggedright\arraybackslash}p{(\columnwidth - 4\tabcolsep) * \real{0.3400}}@{}}
\toprule
\begin{minipage}[b]{\linewidth}\raggedright
Name
\end{minipage} & \begin{minipage}[b]{\linewidth}\raggedright
Equation
\end{minipage} & \begin{minipage}[b]{\linewidth}\raggedright
Explanation
\end{minipage} \\
\midrule
\endhead
\(S_1\) & \(x_{a,t} = \beta x_{a-1,t-1} + \varepsilon\) & Constant \\
\(S_2\) & \(x_{a,t} = \beta x_{a-1,t-1} + \gamma_t + \varepsilon\) &
Annual deviations \\
\(S_3\) & \(x_{a,t} = \beta x_{a-1,t-1} + \nu_c + \varepsilon\) & Cohort
deviations \\
\(S_4\) & \(x_{a,t} = \beta x_{a-1,t-1} + \varepsilon\)

\(x_{0,t} \sim N(x_c, \sigma_c^2)\) & Initial size deviations \\
\(S_5\) &
\(x_{a,t} = \beta x_{a-1,t-1} + \gamma_t + \nu_c + \varepsilon\)
\textbar{} Annua

\(x_{0,t} \sim N(x_c, \sigma_c^2)\) & l, cohort, and initial size
deviations \\
\(S_6\) &
\(x_{a,t} = \beta x_{a-1,t-1} + \gamma_t + \beta_y T_{y-a} + \varepsilon\)
\textbar{} Annual de & viations with temperature effects \\
\(S_7\) &
\(x_{a,t} = \beta x_{a-1,t-1} + \nu_c + \beta_c T_{y-a} + \varepsilon\)
& Cohort deviations with temperature effects \\
\(S_8\) & \(x_{a,t} = \beta x_{a-1,t-1} + \varepsilon\)

\(x_{0,t} \sim N(x_c+\beta_iT_{y-a}, \sigma_c^2) \forall t\) \textbar{}
& Initial size deviations with temperature effects \\
\bottomrule
\end{longtable}

Table 4: Comparison of AIC values achieved by the DFA models applied to
\(L_{\infty}\) time series (first method). The lowest AIC in each area
is highlighted in bold.

\begin{longtable}[]{@{}llllll@{}}
\toprule
Region & ECV & Monterey & Conception & & \\
\midrule
\endhead
model & AICc & AICc & AICc & & \\
------- & ----- & --------- & ---------- & & \\
no covariate & 261 & 215 & 142 & & \\
temp \textless=183m & 273 & 235 & 163 & & \\
temp 184-550m & 274 & 236 & & & \\
temp \textgreater=550m & 275 & & & & \\
all temp & 289 & 252 & & & \\
fishing & 258 & 232 & 157 & & \\
fishing \& temp \textless=183m & 266 & 250 & 184 & & \\
fishing \& temp 184-550m & 267 & 251 & & & \\
fishing \& temp \textgreater=550m & 266 & & & & \\
all covariates & 317 & 273 & 184 & & \\
\bottomrule
\end{longtable}

Table 5: Comparison of AIC values achieved by the DFA models applied to
time series of mean length at juvenile, maturing and mature life stages
(second method). The lowest AIC in each area and life stage is
highlighted in bold.

\begin{longtable}[]{@{}
  >{\raggedright\arraybackslash}p{(\columnwidth - 10\tabcolsep) * \real{0.1667}}
  >{\raggedright\arraybackslash}p{(\columnwidth - 10\tabcolsep) * \real{0.1667}}
  >{\raggedright\arraybackslash}p{(\columnwidth - 10\tabcolsep) * \real{0.1667}}
  >{\raggedright\arraybackslash}p{(\columnwidth - 10\tabcolsep) * \real{0.1667}}
  >{\raggedright\arraybackslash}p{(\columnwidth - 10\tabcolsep) * \real{0.1667}}
  >{\raggedright\arraybackslash}p{(\columnwidth - 10\tabcolsep) * \real{0.1667}}@{}}
\toprule
\begin{minipage}[b]{\linewidth}\raggedright
\end{minipage} & \begin{minipage}[b]{\linewidth}\raggedright
ECV
\end{minipage} & \begin{minipage}[b]{\linewidth}\raggedright
\end{minipage} & \begin{minipage}[b]{\linewidth}\raggedright
\end{minipage} & \begin{minipage}[b]{\linewidth}\raggedright
\end{minipage} & \begin{minipage}[b]{\linewidth}\raggedright
Monterey
\end{minipage} \\
\midrule
\endhead
& age\_juv & A50 & age\_mat & age\_juv & A50 \\
\begin{minipage}[t]{\linewidth}\raggedright
\begin{center}\rule{0.5\linewidth}{0.5pt}\end{center}
\end{minipage} & \begin{minipage}[t]{\linewidth}\raggedright
\begin{center}\rule{0.5\linewidth}{0.5pt}\end{center}
\end{minipage} & \begin{minipage}[t]{\linewidth}\raggedright
\begin{center}\rule{0.5\linewidth}{0.5pt}\end{center}
\end{minipage} & \begin{minipage}[t]{\linewidth}\raggedright
\begin{center}\rule{0.5\linewidth}{0.5pt}\end{center}
\end{minipage} & \begin{minipage}[t]{\linewidth}\raggedright
\begin{center}\rule{0.5\linewidth}{0.5pt}\end{center}
\end{minipage} & \begin{minipage}[t]{\linewidth}\raggedright
\begin{center}\rule{0.5\linewidth}{0.5pt}\end{center}
\end{minipage} \\
model & AICc & AICc & AICc & AICc & AICc \\
\begin{minipage}[t]{\linewidth}\raggedright
\begin{center}\rule{0.5\linewidth}{0.5pt}\end{center}
\end{minipage} & \begin{minipage}[t]{\linewidth}\raggedright
\begin{center}\rule{0.5\linewidth}{0.5pt}\end{center}
\end{minipage} & \begin{minipage}[t]{\linewidth}\raggedright
\begin{center}\rule{0.5\linewidth}{0.5pt}\end{center}
\end{minipage} & \begin{minipage}[t]{\linewidth}\raggedright
\begin{center}\rule{0.5\linewidth}{0.5pt}\end{center}
\end{minipage} & \begin{minipage}[t]{\linewidth}\raggedright
\begin{center}\rule{0.5\linewidth}{0.5pt}\end{center}
\end{minipage} & \begin{minipage}[t]{\linewidth}\raggedright
\begin{center}\rule{0.5\linewidth}{0.5pt}\end{center}
\end{minipage} \\
no covars & 249 & 249 & 224 & 248 & 264 \\
temp \textless=183m & 267 & 267 & 242 & 265 & 282 \\
temp 184-550m & 266 & 267 & 242 & 265 & 282 \\
temp \textgreater=550m & 266 & 267 & 242 & & \\
all temp & 284 & 297 & 262 & 282 & 291 \\
fishing & 265 & 265 & 237 & 265 & 282 \\
fishing \& temp \textless=183m & 284 & 285 & 250 & 287 & 303 \\
fishing \& temp 184-550m & 283 & 285 & 250 & 287 & 303 \\
fishing \& temp \textgreater=550m 282 & 284 & 250 & & & \\
all covars & 318 & 334 & 278 & 308 & 317 \\
\bottomrule
\end{longtable}

\hypertarget{figures}{%
\section{Figures}\label{figures}}

\begin{figure}
\caption{Effects of bottom temperature anomaly on the von Bertelanffy growth parameter $k$ for each species.}\label{temp_effect_k}
\begin{center}
\includegraphics[height = 0.8\textheight]{../results/ss_lvb_temp/temp_effect_by_species.pdf}
\end{center}
\end{figure}

\begin{figure}
\caption{Annual spawning biomass, fully-selected fishing mortality rate, and recruitment for Petrale sole.}\label{SSB_F_R}
\begin{center}
\includegraphics[height = 0.8\textheight]{../results/ssm_temp/petrale_SSB_F_R.pdf}
\end{center}
\end{figure}

\begin{figure}
\caption{Annual SSB$_{40}$ and $F_{40}$, reference points for Petrale sole.}\label{BRPs}
\begin{center}
\includegraphics[height = 0.8\textheight]{../results/ssm_temp/petrale_BRPs.pdf}
\end{center}
\end{figure}

\end{document}
